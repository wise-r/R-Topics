
\documentclass[12pt]{article}

\usepackage{lipsum}
\usepackage{amsmath}
\usepackage{amsthm}
\usepackage{amssymb}
\usepackage{bm}
\usepackage{verbatim}
\usepackage{graphicx}
\usepackage[T1]{fontenc}
\usepackage{multirow} 
\usepackage[body={16cm,24cm}, top=3cm]{geometry}
\geometry{papersize={21.59cm,27.94cm}}

\parindent=0in
\pagestyle{headings}

\begin{document}

%\title{\textbf{Macro II Problem Set 6}}
\title{
 \textbf{Macro II Problem Set 6 }\\
  \large Welfare Analysis of Practical Monetary Policy Rules\\
   within the Basic New Keyesian Model}

\author{
  QK\\
  \texttt{277201411527XX}
  \and
 CSS\\
  \texttt{277201411527XX}
}
\maketitle

\section{Part I. A Taylor-type Interest Rate Rule.}  %% PART I
\subsection{\dots Write out the entries in the new vector $\tilde{B}_{t}$} %problem 1
\normalsize{\textbf{A.N.S:}}
Assuming that variations in the technology parameter $a_t$ represent the only driving force in the economy, and are described by a stationary AR(1) process with autoregressive coefficient $\rho_a$, the following equality holds:
\begin{equation}
\begin{aligned}
\hat{r}_{t}^{n}-v_{t} &= -\sigma\psi^{n}_{ya}(1-\rho_{a})a_{t}-\phi_{y}\psi^{n}_{ya}a_{t}\\
&=-\psi^{n}_{ya}[\sigma(1-\rho_{a})+\phi_{y}]a_{t}
\end{aligned}
\end{equation}
Then, we have:
\begin{equation}
\begin{aligned}
B_{T}(\hat{r_{t}}^{n}-v_{t})&=-B_{T}(\psi^{n}_{ya}[\sigma(1-\rho_{a})+\phi_{y}])a_{t}\\
& = \tilde{B}_{t}a_{t}
\end{aligned}
\end{equation}
\[ \Longrightarrow   \tilde{B}_{t}= -B_{T}(\psi^{n}_{ya}[\sigma(1-\rho_{a})+\phi_{y}]) \]

\subsection{Show how to compute $var(a_t)$?, And what is the value of $var(a_t)$? }%problem 2
\textbf{A.N.S:} For $a_t=\rho_{a}a_{t-1}+\varepsilon_{t}^{a}$, 
\begin{equation*}
Var(a_t) = \rho_a^2 Var(a_{t-1})+Var(\varepsilon_{t}^{a})
\end{equation*}
Assume $Var(a_t) = Var(a_{t-1})$, we can find:
\begin{equation}
Var(a_{t}) = \frac{Var(\varepsilon_{t}^{a})}{1- \rho_a^2 } = \frac{\sigma_{a}^2}{1- \rho_a^2 } \label{eq3}
\end{equation}
Substitute $\sigma_{a}^2 = 0.00712^2$ and $\rho_a = 0.9$ into equation (\ref{eq3}), we can compute 
\[Var(a_{t})= 2.6681\times 10^{-4} \]

\subsection{Please show how to compute $Var(\tilde{y}_{t})$ and $Var(\pi_{t})$ as a function of $var(a_t)$ } %problem 3
\textbf{A.N.S:} Consider the DIS and NKPC system:
\begin{equation}
\left[
\begin{array}{c}
 \tilde{y}_{t}\\
 \pi_{t}
\end{array}
\right] = A_T
\left[
\begin{array}{c}
E_{t}\{\tilde{y}_{t+1}\}\\
E_{t}\{ \pi_{t+1}\}
\end{array}
\right]+ \tilde{B}_{T}a_t\label{eq4}
\end{equation}
where ,
\begin{gather*}
 A_T = 
\left(
\begin{array}{cc}
 \sigma & 1-\beta\phi_{\pi}      \\
  \kappa\sigma & \kappa+\beta(\sigma+\phi_y)        
\end{array}
\right)\Omega\\
 \tilde{B}_{t}= -B_{T}(\psi^{n}_{ya}[\sigma(1-\rho_{a})+\phi_{y}])\\
B_{T} =\left[
\begin{array}{c}
1\\
\kappa
\end{array}
\right]\Omega\\
\kappa = \lambda(\sigma+\frac{\varphi+\alpha}{1-\alpha})\\
\lambda = \frac{(1-\theta)(1-\beta\theta)}{\theta}\Theta\\
\Theta = \frac{1-\alpha}{1-\alpha-\alpha\varepsilon}
\end{gather*}
Rewrite Equation(\ref{eq4}) in a recursive form:
\begin{equation}
\begin{aligned}
\left[
\begin{array}{c}
 \tilde{y}_{t}\\
 \pi_{t}
\end{array}
 \right] &= A_T 
 \{
 A_T
\left[
\begin{array}{c}
E_{t}\{\tilde{y}_{t+2}\}\\
E_{t}\{ \pi_{t+2}\}
\end{array}
\right]+ \tilde{B}_{T}a_{t+1}
 \}+ \tilde{B}_{T}a_t\\
 & = (A_T)^2\left[
\begin{array}{c}
E_{t}\{\tilde{y}_{t+2}\}\\
E_{t}\{ \pi_{t+2}\}
\end{array}
\right]+A_T\rho_a\tilde{B}_{T}a_{t}+\tilde{B}_{T}a_t\\
&\qquad\vdots\qquad\vdots\qquad\vdots\\
&=(A_T)^{\infty}\left[
\begin{array}{c}
E_{t}\{\tilde{y}_{\infty}\}\\
E_{t}\{ \pi_{\infty}\}
\end{array}
\right]+\sum_{i=0}^{\infty}(A_T\rho_a)^{i}\tilde{B}_{T}a_{t}\\
& = (I_2 - \rho_a A_T)^{-1} \tilde{B}_{T}a_{t}
\end{aligned}
\end{equation}
Denote $2\times 1$ vector $ C = (I_2 - \rho_a A_T)^{-1} \tilde{B}_{T} $, thus,
\begin{gather}
  var(\tilde{y}_{t}) = [C(1)]^2 var(a_t)\\
  var(\pi_{t}) = [C(2)]^2 var(a_t)
\end{gather}
where $C(i)$ is the $i-th$ element of $C$



%We can rewrite equation (\ref{eq4}) as :
%\begin{gather}
%\tilde{y}_{t} =  \sigma\Omega E_{t}\{\tilde{y}_{t+1}\}+ (1-\beta\phi_{\pi} )\Omega E_{t}\{ \pi_{t+1}\}-\Omega(\psi^{n}_{ya}[\sigma(1-\rho_{a})+\phi_{y}])a_{t}\\
%\pi_{t} =  \kappa\sigma\Omega E_{t}\{\tilde{y}_{t+1}\} + (\kappa+\beta(\sigma+\phi_y))\Omega E_{t}\{ \pi_{t+1}\}-\Omega\kappa(\psi^{n}_{ya}[\sigma(1-\rho_{a})+\phi_{y}])a_{t}
%\end{gather}
%Assume $Var(\tilde{y}_{t})=Var(E_{t}\{\tilde{y}_{t+1}\})$ and $Var( \pi_{t}) = Var( \pi_{t+1})$, the variance of $\tilde{y}_{t}$ and $\pi_{t}$ satisfy:
%\begin{gather}
%Var(\tilde{y}_{t}) =  (\sigma\Omega)^2 Var(\tilde{y}_{t})+ [(1-\beta\phi_{\pi} )\Omega]^2 Var( \pi_{t})+[\Omega(\psi^{n}_{ya}[\sigma(1-\rho_{a})+\phi_{y}])]^2Var(a_{t})\\
%Var( \pi_{t})= ( \kappa\sigma\Omega)^2 Var(\tilde{y}_{t})  + [(\kappa+\beta(\sigma+\phi_y))\Omega]^2 Var( \pi_{t})+[\Omega\kappa(\psi^{n}_{ya}[\sigma(1-\rho_{a})+\phi_{y}])]^2Var(a_{t})
%\end{gather}
Use Matlab , we can easily solve the system:
\begin{gather*}
 var(\tilde{y}_{t}) =  1.5645\times 10^{-5} \\
 var( \pi_{t}) =1.5324\times 10^{-4}\\
\left[
\begin{array}{c}
Var(\tilde{y}_{t}) \\
 Var( \pi_{t})
\end{array}
\right] = \left[
\begin{array}{c}
 1.5645\times 10^{-5} \\
1.5324\times 10^{-4}
\end{array}
\right]
  \end{gather*}
\newpage
\subsection{Compute the average welfare loss per period.} %% Problem 4
\textbf{A.N.S:}The average welfare loss per period in terms of variations of output gap and inflation is:
\begin{equation}
L = \frac{1}{2}\big[ (\sigma+\frac{\varphi+\alpha}{1-\alpha})Var(\tilde{y}_{t})+\frac{\varepsilon}{\lambda} Var( \pi_{t})\big]\label{eq9}
\end{equation}
Substituting $Var(\tilde{y}_{t})$ and $Var( \pi_{t})$ into  equation (\ref{eq9}), we can solve the welfare loss:
\begin{equation*}
L = 0.0041
\end{equation*}

\subsection{Vary the values of policy reaction parameters and generate a table as follows, and fill the blank entries:}
\textbf{A.N.S:} Repeated the problem 1.3,1.4 procedure with respect to the given $\phi_{\pi}$ and  $\phi_{y}$. 
{\centering
\begin{table}[h]
\centering
\begin{tabular}{lllll}
\hline
 & \multicolumn{4}{l}{Taylor Rule} \\ \hline
$\phi_{\pi}$ &    \textbf{1.5}    &  \textbf{1.5}      & \textbf{1.5}      &  \textbf{5}     \\
 $\phi_{y}$&    \textbf{1 }   &   \textbf{0.125}     &   \textbf{0}    & \textbf{ 0}     \\
 $\sigma(\tilde{y})$&   $3.6408 \times 10^{-5}$   &  $3.0547\times 10^{-6}$      &  $6.8208\times 10^{-7}$     & $1.5955\times 10^{-8}$      \\ 
 $\sigma(\pi)$&   $3.5662 \times 10^{-4}$    &  $2.9921  \times 10^{-5}$      &  $6.6810\times 10^{-6} $  & $1.5628\times 10^{-7}$       \\ 
  $\sigma(L)$&   $0.0095$  &     $7.9396 \times 10^{-4}$    &   $1.7728\times 10^{-4}$     &  $4.1469\times 10^{-6}$    \\   \hline
\end{tabular}
\end{table}}
\newpage
\section{Part II.  A Money Growth Rule.}  %% PART 2
\subsection{Compute $\hat{r}_t^n$ and  $\hat{y}_t^n$ }
From Chapter 3 Gali(2008) \footnote{Gali, Jordi. Monetary Policy, inflation, and the Business Cycle: An introduction to the new Keynesian Framework. Princeton University Press, 2008.}, we know the following relations:
\begin{gather}
 \hat{y}_t^n = \psi_{ya}^n a_t\\
 \hat{r}_t^n = -\sigma\psi_{ya}^n(1-\rho_{a})a_t
\end{gather}
where $\psi_{ya}^n =\frac{(1+\varphi)}{\varphi+\alpha+\sigma(1-\alpha)} $.
Hence, we have:
\begin{gather}
  var(\hat{y}_t^n) =  (\psi_{ya}^n)^2 var(a_t) =  2.6681\times 10^{-4}\\
  var(\hat{r}_t^n) = [\sigma\psi_{ya}^n(1-\rho_{a})]^2  var(a_t) = 2.6681\times 10^{-6}
\end{gather}
\subsection{ Compute $var(\Delta\zeta_t)$}
Since the  exogenous money demand shock$\zeta_t$ is an AR(1) process,
\begin{equation}
var(\Delta\zeta_t) = \frac{\sigma(\zeta)}{1-\rho_{\zeta}^2} = 6.2016\times 10^{-5}
\end{equation}

\subsection{Show how to compute $\big[ var(\tilde{y}_t)\quad var(\pi_t) \quad var(\Delta\zeta_t) \big]'$}
Following page 56 of Gali(2008),  we know:
\begin{equation*}
A_{M,0}\equiv\left[
\begin{array}{ccc}
 1+\sigma\eta & 0  & 0  \\
  -\kappa& 1  & 0   \\
  0 & -1  &  1  
\end{array}
\right], \quad 
A_{M,1}\equiv\left[
\begin{array}{ccc}
 \sigma\eta & \eta  & 1  \\
  0& \beta  & 0   \\
  0 & 0  &  1  
\end{array}
\right], \quad 
B_{M}\equiv\left[
\begin{array}{ccc}
 \eta & -1 & 0  \\
 0& 0  & 0   \\
  0 & 0 &  -1  
\end{array}
\right]
\end{equation*}
Split the linear matrix equation into a three equations system, we have:
\begin{gather}
 (1+\sigma\eta)\tilde{y}_{t} = \sigma\eta E_{t}(\tilde{y}_{t+1})+\hat{l}_{t}^{+}+\eta\hat{r_{t}}^{n} - \hat{y_{t}}^{n}\\
 -\kappa\tilde{y}_{t} + \pi_{t} = \beta E_{t}(\pi_{t+1})\\
 - \pi_{t} +\hat{l}_{t-1}^{+} = \hat{l_{t}}^{+} - \Delta\zeta_{t} \label{eq16}
\end{gather}
Taking expectation w.r.t equation (\ref{eq16}) and rearranging the equation we have,
\begin{equation}
 \hat{l_{t}}^{+} = E_{t}(\pi_{t+1})+E_{t}( \hat{l}_{t+1}^{+})-\rho_{\zeta} \Delta\zeta_{t}
\end{equation}
Equations (14), (15) and (17)  can be summarized compactly by the system:

\begin{equation}
\left[
\begin{array}{ccc}
 1+\sigma\eta & 0  & 0  \\
  -\kappa& 1  & 0   \\
  0 & 0  &  1  
\end{array}
\right]
\left[
\begin{array}{c}
\tilde{y}_{t}\\
\pi_{t}\\
\hat{l_{t}}^{+}
\end{array}
\right]=
\left[
\begin{array}{ccc}
 \sigma\eta & \eta  & 1  \\
  0& \beta  & 0   \\
  0 & 1  &  1  
\end{array}
\right]
\left[
\begin{array}{c}
E_{t}(\tilde{y}_{t+1})\\
E_{t}(\pi_{t+2})\\
E_{t}(\hat{l}_{t+1}^{+})
\end{array}
\right]
+ 
\left[
\begin{array}{cc}
 -1.4 & -0.6 \\
  0 & 0   \\
  0 & -0.6
\end{array}
\right]
\left[
\begin{array}{ccc}
 a_t    \\
   \Delta\zeta_{t} \\  
\end{array}
\right]
\end{equation}

Denote
 \begin{equation}
A \left[
\begin{array}{c}
\tilde{y}_{t}\\
\pi_{t}\\
\hat{l_{t}}^{+}
\end{array}
\right]=B\left[
\begin{array}{c}
E_{t}(\tilde{y}_{t+1})\\
E_{t}(\pi_{t+2})\\
E_{t}(\hat{l}_{t+1}^{+})
\end{array}
\right]+C\left[
\begin{array}{c}
 a_t    \\
   \Delta\zeta_{t} \\  
\end{array}
\right]
\end{equation}
Solve the system, we get 
\begin{equation}
\left[
\begin{array}{c}
\tilde{y}_{t}\\
\pi_{t}\\
\hat{l_{t}}^{+}
\end{array}
\right]=D \left[
\begin{array}{c}
 a_t    \\
   \Delta\zeta_{t} \\  
\end{array}
\right]
\end{equation}
Substituting the known parameters we can solve,
 \begin{equation}
D =
\left[
\begin{array}{cc}
  0.10496 & 0.28902   \\
 0.12277 &    0.5773 \\
  1.105&   −0.63406  
\end{array}
\right]
\end{equation}
We have
 \begin{gather}
 var(\tilde{y}_{t}) = D(1,1)^2 var(a_t) + D(1,2)^2 var( \Delta\zeta_{t})\\
 var(\pi_{t}) = D(2,1)^2 var(a_t) + D(2,2)^2 var( \Delta\zeta_{t})
\end{gather}

\subsection{Compute the values of $\big[ var(\tilde{y}_t)\quad var(\pi_t) \quad var(\Delta\zeta_t) \big]'$ }
Substituting the known parameters into equation (22) an (23), we find:
\begin{equation*}
\begin{aligned}
var(\tilde{y}_{t}) = 8.1197 \times 10^{-6}\\
var(\pi_{t}) = 2.469 \times 10^{-5}\\
\end{aligned}
\end{equation*}
\subsection{Compute the average welfare loss per period.}
Frome equation (8), we find 
L = 0.0075
\newpage

\section{Figure}

\begin{figure}
\centering
   \includegraphics[height=5cm, width=10cm]{img/doge.jpg}
  \caption{Doge}
  \label{fig:stock}
\end{figure}



\section{Codes}

\subsection{Codes for Part I}
\begin{verbatim}
%% ps6I_par.m This file stores the given parameter value.
beta = .99;
sigma = 1;
phi = 1;
alpha = 1/3;
epsilon = 6;
theta = 2/3;
phi_pi = 1.5;
phi_y = 0.5; % change this value to obtain various outcome
rho_v = 0.5;
rho_a = 0.9;

var_a = 0.00712^2/(1-0.9^2)
ps6_go

%% ps6I_go.m  This file undergo the procedure to calculate the value required
%Caculating parameters 
phi_ya_n=(1+phi)/(phi+alpha+sigma*(1-alpha));
kappa=(sigma+(phi+alpha)/(1-alpha))*(1-theta)*(1-beta*theta)/theta*(1-alpha)/(1-alpha+alpha*epsilon);
kappa = lambda*(sigma+(phi+alpha)/(1-alpha));
Omega = 1/(sigma+phi_y+kappa*phi_pi)
AT = [sigma*Omega, (1-beta*phi_pi)*Omega; (kappa*sigma)*Omega, (kappa+beta*(sigma+phi_y))*Omega]
BT = [1*Omega; kappa*Omega]
residual_par = -psi_nya*(sigma*(1-rho_a)+phi_y);
BT_tilde = [-psi_nya*(sigma*(1-rho_a)+phi_y)*Omega; -kappa*psi_nya*(sigma*(1-rho_a)+phi_y)*Omega]
I = eye(2);
C = (I - rho_a*AT)^(-1)*BT_tilde
var_y = C(1)^2*var_a
var_pi = C(2)^2*var_a
WL = 0.5*((sigma+((phi+alpha)/(1-alpha)))*var_y + (epsilon/lambda)*var_pi)
\end{verbatim}

\subsection{Codes for Part II}
\begin{verbatim}
%% ps6II_par.m This file stores the given parameter value.
beta = .99;
sigma = 1;
phi = 1;
alpha = 1/3;
epsilon = 6;
theta = 2/3;
phi_pi = 1.5;
phi_y = 0.5; % change this value to obtain various outcome
rho_v = 0.5;
rho_a = 0.9;

var_zeta = var_zetaepsilon^2/(1-rho_zeta^2)
var_a = 0.00712^2/(1-0.9^2)
ps6_go

%% ps6II_go.m  This file undergo the procedure to calculate the value required
%Caculating parameters 

var_ny = (psi_nya)^2*var_a
var_nr = (sigma*psi_nya*(1-rho_a))^2*var_a

phi_ya_n=(1+phi)/(phi+alpha+sigma*(1-alpha));
kappa=(sigma+(phi+alpha)/(1-alpha))*(1-theta)*(1-beta*theta)/theta*(1-alpha)/(1-alpha+alpha*epsilon);
kappa = lambda*(sigma+(phi+alpha)/(1-alpha));
Omega = 1/(sigma+phi_y+kappa*phi_pi)
AM0=[1+sigma*eta  0  0  
  kappa 1  0   
  0  0   1  ]
  AM1 =[sigma*eta  eta   1  
  0 beta  0   
  0  0   1  ]
 BM = [ eta  -1  0  
 0 0   0   
  0 0  -1  ]
  
  A = 1+sigma*eta  0  0  
 - kappa 1  0   
  0  -1  1  ]
  B= [sigma*eta  eta   1  
  0 beta  0   
  0  1   1  ]
  C= [-1.4  -0.6 
          0  0   
         0  -0.6]     
 D = [  0.10496 0.28902 
 0.12277 0.5773 
1.105 0.63406]
var_y = D(1,1)^2*var_a+ D(1,1)^2*var_zeta
var_pi = D(2,1)^2*var_a + D(2,2)^2*var_zeta
 
WL = 0.5*((sigma+((phi+alpha)/(1-alpha)))*var_y + (epsilon/lambda)*var_pi)

\end{verbatim}


\end{document}
